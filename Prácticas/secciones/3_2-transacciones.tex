\section{Identificación de transacciones}
\subsubsection{Jugadores/Entrenadores: Clara María Romero Lara}
Consultar entrenador y parejas que entrena en una edición \texttt{mostrarEntrenador}

\begin{lstlisting}[language=sql]
CREATE OR REPLACE PROCEDURE mostrarEntrenador(anio IN NUMBER, idEntr IN NUMBER, salida OUT CLOB)
IS
	CURSOR cEntrena IS SELECT * FROM Entrena WHERE (anio = Año) AND (idEntr=idEntrenador);
	coach Entrena%ROWTYPE;
	coach_row CLOB;
BEGIN
	salida := 'Entrenador '||idEntr||' en la edicion '||anio||' entrena:'||CHR(10);
	OPEN cEntrena;
	FETCH cEntrena INTO coach;

	WHILE (cEntrena%FOUND) LOOP
		coach_row := '   Pareja: ' || coach.idJugador1 || ' ' || coach.idJugador2;
		salida := salida || coach_row || CHR(10);
		FETCH cEntrena INTO coach;
	END LOOP;

	CLOSE cEntrena;
END;
/
\end{lstlisting}

Previa su ejecución, deberíamos:
\begin{itemize}
	\item Insertar una edición \texttt{insertarEdicion}
	\item Insertar al menos dos jugadores \texttt{insertarJugador}
	\item Registrar una pareja \texttt{registrarPareja}
	\item Registrar una pareja en una edición \texttt{registrarInscrita}
	\item Insertar un entrenador \texttt{insertarEntrenador}
	\item Asignar un entrenador a una pareja en una edición \texttt{registrarEntrena}
\end{itemize}

\subsubsection{Pistas/Partidos: Javier Expósito Martínez}
Insertar un nuevo partido \texttt{insertarPartido}.

Se registra un partido en una pista y se introduce una tupla en la tabla \texttt{Jugado}. Es decir, en las tablas \texttt{SeJuegaEn} y \texttt{Jugado}
sólo se pueden insertar tuplas si insertamos un partido.

\begin{lstlisting}[language=sql]
CREATE OR REPLACE PROCEDURE InsertarPartido(idPartido NUMBER, fecha TIMESTAMP, resultado VARCHAR2, idEdicion NUMBER, idPista NUMBER) IS
BEGIN
	INSERT INTO Partido(idPartido,fecha,resultado) VALUES (idPartido,fecha,resultado);
	 INSERT INTO SeJuegaEN(idPartido,idPista) VALUES (idPartido,idPista);
	 INSERT INTO Jugado(idPartido,Año) VALUES (idPartido,idEdicion);
END;
/
\end{lstlisting}

Previa su ejecución, deberíamos:
\begin{itemize}
	\item Insertar una edición \texttt{InsertarEdicion}
	\item Insertar una pista \texttt{InsertarPista}
\end{itemize}

\pagebreak

\subsubsection{Patrocinadores/Colaboradores: Leire Requena Garcia}
Mostrar el dinero total obtenido en una edición \texttt{mostrarRecaducacion}.

\begin{lstlisting}[language=sql]
CREATE OR REPLACE PROCEDURE mostrarRecaudacion(edicion IN NUMBER, recaudacion OUT NUMBER) IS
	RecaudacionColab INT;
	RecaudacionPatr INT;
BEGIN
	SELECT SUM(Dinero_aportado)
	INTO RecaudacionColab
	FROM Colabora
	WHERE Año=edicion;

	SELECT SUM(Dinero_aportado)
	INTO RecaudacionPatr
	FROM Patrocina
	WHERE Año=edicion;

	recaudacion := RecaudacionPatr + RecaudacionColab;
END;
/
\end{lstlisting}

Previa su ejecución, deberíamos:
\begin{itemize}
	\item Insertar una edición \texttt{insertarEdicion}
	\item Insertar una entidad \texttt{insertarEntidad}
	\item Registrar al menos un colaborador y/o un patrocinador \texttt{registrarColaborador, registrarPatrocinador}
\end{itemize}

\subsubsection{Personal/Horarios: Laura Sánchez Sánchez}
Mostrar el personal que no trabaja en una edición \texttt{mostrarPersonal}.

\begin{lstlisting}[language=sql]
CREATE OR REPLACE PROCEDURE mostrarPersonal (anio IN NUMBER, salida OUT CLOB)
IS
	CURSOR cPersonal IS
	(SELECT * FROM Personal
		WHERE NOT EXISTS(
				SELECT IDPERSONAL FROM Trabaja
				WHERE (anio=Trabaja.Año AND Trabaja.idPersonal = Personal.idPersonal)
	));

	persona Personal%ROWTYPE;
BEGIN
	OPEN cPersonal;
	FETCH cPersonal INTO persona;
	WHILE (cPersonal%FOUND) LOOP
		salida := salida ||'ID: ' || persona.idPersonal || ' Nombre y apellidos: ' ||
		persona.nombreP || ' ' || persona.apellidosP || ' Email: ' || persona.emailp ||
		CHR(10);
		FETCH cPersonal INTO persona;
	END LOOP;

	CLOSE cPersonal;
END;
/
\end{lstlisting}

\pagebreak

Previa su ejecución, deberíamos:
\begin{itemize}
	\item Insertar una edición \texttt{insertarEdicion}
	\item Insertar personal \texttt{insertarPersonal}
	\item Registrar personal como trabajador en edición \texttt{registrarTrabajador}
\end{itemize}

\subsubsection{Materiales/Pedidos: Inés Nieto Sánchez}
Asignar un pedido a un trabajador \texttt{InsertarAsignacionRecogida}.

\begin{lstlisting}[language=sql]
CREATE OR REPLACE PROCEDURE InsertarAsignacionRecogida(numeroPedido INT,
idPersonal INT, Año INT,idPista INT, fechainicio TIMESTAMP, fechafin TIMESTAMP,
fecha TIMESTAMP)
IS
BEGIN
	INSERT INTO Recoge(numero_pedido,idpersonal,año,idpista,fechainicio,fechafin,fecha)
	VALUES(numeroPedido,idpersonal,año,idpista,fechainicio,fechafin,fecha);
END;
/
\end{lstlisting}

Previa su ejecución, deberíamos:
\begin{itemize}
	\item Insertar una edición \texttt{insertarEdicion}
	\item Insertar personal \texttt{insertarPersonal}
	\item Registrar personal como trabajador en edición \texttt{registrarTrabajador}
	\item Asignar trabajador a pista \texttt{InsertarAsigna}
	\item Insertar material \texttt{InsertarMaterial}
	\item Insertar pedido \texttt{InsertarPedido}
\end{itemize}

\subsubsection{Disparadores y procedimientos adicionales}
Además de estas funcionalidades y sus respectivas para una correcta ejecución,
hemos incluído ciertas restricciones y funcionalidades del sistema tales como:
\begin{itemize}
	\item Triggers:
	\begin{itemize}
		\item No se permite ser patrocinador y colaborador en la misma edición
		\item No se permite a una pareja inscrita tener la misma posición ranking que otra
	\end{itemize}
	\item Procedures:
	\begin{itemize}
		\item Registrar qué patrocinador proporciona un material en una edición \texttt{InsertarProporciona}
		\item Registrar los materiales que componen un pedido \texttt{InsertarArticuloPedido}
		\item Modificar salario de un trabajador \texttt{modificarSalario}
		\item Modificar dinero aportado por un colaborador o patrocinador \texttt{modificarDineroPatr}, \texttt{modificaDineroColab}
	\end{itemize}
\end{itemize}

Para ver el código de los procedimientos y triggers adicionales ir al \hyperref[cap:extras]{Apéndice A}
