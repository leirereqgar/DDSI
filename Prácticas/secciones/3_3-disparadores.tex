\section{Disparadores}

\subsection{Un jugador no puede pertenecer a dos parejas distintas en la misma edición}
Este trigger saltará si se intenta inscribir una pareja con un jugador que ya esté
inscrito con otra en la misma edición.

Con el cursor \texttt{cJugador} recorremos todas las parejas inscritas en el
mismo año en el que nos vamos a inscribir. Si alguna de estas tiene el mismo
\texttt{idJugador1} o \texttt{idJugador2} registrado, saltará una excepción y
la pareja no se inscribirá en la edición.

\begin{lstlisting}[language=sql]
CREATE OR REPLACE TRIGGER jugador_unico
	BEFORE
	INSERT ON Inscrita
	FOR EACH ROW
DECLARE
	CURSOR cJugador IS SELECT * FROM Inscrita WHERE (:new.Año = Año);
	jugadores Inscrita%ROWTYPE;
BEGIN
	OPEN cJugador;
	FETCH cJugador INTO jugadores;
	WHILE (cJugador%FOUND) LOOP
		IF (:new.idJugador1 = jugadores.idJugador1 OR :new.IdJugador1 = jugadores.idJugador2)
			THEN raise_application_error(-20600, :new.idJugador1 || 'JUGADOR 1: No se puede pertenecer a más de una pareja en la misma edicion');
		END IF;

		IF (:new.idJugador2 = jugadores.idJugador1 OR :new.IdJugador2 = jugadores.idJugador2)
			THEN raise_application_error(-20600, :new.idJugador2 || 'JUGADOR 2: No se puede pertenecer a más de una pareja en la misma edicion');
		END IF;
		FETCH cJugador INTO jugadores;
	END LOOP;

	CLOSE cJugador;
END;
\end{lstlisting}

\subsection{No se pueden jugar más de dos partidos en la misma pista}
Este disparador se encargará de comprobar si se puede o no insertar un partido
y pista en la tabla \texttt{SeJuegaEn}, teniendo en cuenta la restricción.

El principal problema de este disparador es que la fecha en la que se juega el
partido no está en la tabla de \texttt{SeJuegaEn} donde actúa nuestro disparador,
sino en la tabla de \texttt{Partido}, luego para poder relacionarla hemos creado
una vista \texttt{PistaPartidoFecha} que muestre la fecha, el identificador del
partido y el identificador de la pista en los casos donde se juegue un partido en
la misma pista y el mismo día:

\begin{lstlisting}[language=sql]
CREATE VIEW PistaPartidoFecha
AS (SELECT DISTINCT SeJuegaEn.idPista,SeJuegaEn.idPartido,Partido.fecha
	FROM SeJuegaEn, SeJuegaEn cjuega, Partido, Partido par
	WHERE(
		((Partido.fecha - par.fecha)<1 AND Partido.idPartido<>par.idPartido)
		AND
		(SeJuegaEn.idPista=cjuega.idPista AND SeJuegaEn.idPartido<>cjuega.idPartido)
		AND
		(Partido.idPartido=SeJuegaEn.idPartido)
));
\end{lstlisting}

Para el disparador necesitamos dos cursores: uno para coger las fechas de los
partidos (\texttt{cPartidoFecha}) y otro para recoger solo los partidos
(\texttt{cPartido}), y una variable donde guardemos el número de partidos que
hay en la misma pista el mismo día.

El número de partidos jugados en la misma fecha y pista no puede ser mayor o
igual que dos: \texttt{numPartidos} se incrementa cada vez que el cursor encuentra
partidos no repetidos que se juegan en la misma pista y cuya fecha tiene menos
de 24 horas de diferencia. Cuando \texttt{numPartidos} es mayor o igual a dos, lanza
una excepción y el partido no se inserta.

\begin{lstlisting}[language=sql]
CREATE OR REPLACE TRIGGER partido_unico
	BEFORE
	INSERT ON SeJuegaEn
	FOR EACH ROW
DECLARE
	CURSOR cPartidoFecha IS SELECT * FROM PistaPartidoFecha;
	CURSOR cPartido IS SELECT * FROM Partido;
	datos PistaPartidoFecha%ROWTYPE;
	datosPartido Partido%ROWTYPE;
	numPartidos number;
BEGIN
	OPEN cPartidoFecha;
	OPEN cPartido;
	SELECT (SELECT COUNT(*) FROM PistaPartidoFecha) INTO numPartidos FROM dual;
	FETCH cPartidoFecha INTO datos;
	FETCH cPartido INTO datosPartido;

	WHILE(cPartidoFecha%FOUND)LOOP
		WHILE(cPartido%FOUND)LOOP
			IF(numPartidos >= 2 AND :new.idPista = datos.idPista AND :new.idPartido <> datos.idPartido AND
				(:new.idPartido = datosPartido.idPartido AND (datosPartido.fecha-datos.fecha) < 1))
				THEN raise_application_error(-20600, :new.idPartido || 'No se jugar más de dos partidos en la misma pista en un mismo dia');
			END IF;

			FETCH cPartido INTO datosPartido;
		END LOOP;

		FETCH cPartidoFecha INTO datos;
	END LOOP;

	CLOSE cPartido;
	CLOSE cPartidoFecha;
END;
\end{lstlisting}

\pagebreak

\subsection{No se puede modificar el dinero aportado por un colaborador o patrocinador si ya ha empezado el torneo}
En esta ocasión hemos creado dos triggers porque necesitaábamos uno para la tabla
\texttt{Colaborador} y otro para la tabla \texttt{Patrocinador}. Ambos triggers
se dispararán si se intenta modificar el dinero aportado en una edición si ya ha
empezado el torneo.

Con el cursor \texttt{cPrimerPartido} buscamos en la tabla \texttt{Partido} la
tupla que tenga la fecha más antigua dentro de la edición y la guardamos en
\texttt{earliest}.

Si \texttt{earliest} es menor que la fecha actual, es decir, es más antigua; se
cancela la modificación y salta el mensaje de error.

\begin{lstlisting}[language=sql]
CREATE OR REPLACE TRIGGER dinero_tras_inicio_colab
	BEFORE UPDATE OF Dinero_aportado ON Colabora
	FOR EACH ROW
DECLARE
	CURSOR cPrimerPartido IS
	SELECT fecha FROM Partido WHERE fecha IN (SELECT MIN(fecha) FROM Partido
	WHERE (EXTRACT(YEAR FROM fecha) = :old.Año) GROUP BY to_char(fecha,'YYYY')) GROUP BY fecha;

	earliest DATE;
BEGIN
	OPEN cPrimerPartido;
	FETCH cPrimerPartido INTO earliest;

	IF (earliest < CURRENT_DATE)
		THEN raise_application_error(-20600, :new.Dinero_aportado || 'No se puede modificar el dinero aportado si ya ha comenzado el torneo');
	END IF;

	CLOSE cPrimerPartido;
END;
/

CREATE OR REPLACE TRIGGER dinero_tras_inicio_patr
	BEFORE UPDATE OF Dinero_aportado ON Patrocina
	FOR EACH ROW
DECLARE
	CURSOR cPrimerPartido IS
	SELECT fecha FROM Partido WHERE fecha IN (SELECT MIN(fecha) FROM Partido
	WHERE (EXTRACT(YEAR FROM fecha) = :old.Año) GROUP BY to_char(fecha,'YYYY')) GROUP BY fecha;

	earliest DATE;
BEGIN
	OPEN cPrimerPartido;
	FETCH cPrimerPartido INTO earliest;

	IF (earliest < CURRENT_DATE)
		THEN raise_application_error(-20600, :new.Dinero_aportado || 'No se puede modificar el dinero aportado si ya ha comenzado el torneo');
	END IF;

	CLOSE cPrimerPartido;
END;
/
\end{lstlisting}

\subsection{No se puede modificar el salario de un trabajador si ya ha realizado algún trabajo}
Este trigger se disparará al intentar modificar el salario de un trabajador
cuando ya haya realizado algún trabajo.

Con el cursor \texttt{cAsigna} recorreremos todas las \texttt{FechaInicio} (del
trabajador que estemos intentando modificar) de la tabla \texttt{Asigna} y si
hay una tupla que tenga una fecha anterior a la actual, y que además corresponda
a la edición en la que estamos modificando, se lanza el mensaje de error del trigger.

\begin{lstlisting}[language=sql]
CREATE OR REPLACE TRIGGER modificacionSalario
	BEFORE
	UPDATE OF sueldo ON Trabaja
	FOR EACH ROW
DECLARE
	CURSOR cAsigna IS SELECT FechaInicio FROM Asigna WHERE (idPersonal = :new.idPersonal);
	fechas_asigna TIMESTAMP;

BEGIN
	OPEN cAsigna;
	FETCH cAsigna INTO fechas_asigna;
	WHILE (cAsigna%FOUND) LOOP
		IF (CAST(sysdate AS TIMESTAMP) > fechas_asigna AND (EXTRACT(YEAR FROM fechas_asigna) >= :new.Año)) THEN
			RAISE_APPLICATION_ERROR(-20600, :new.sueldo || '***EXCEPTION**** No se puede modificar el sueldo ya que ya ha empezado su trabajo');
		END IF;
		dbms_output.put_line( fechas_asigna );
		FETCH cAsigna INTO fechas_asigna;
	END LOOP;

	CLOSE cAsigna;
END;
/
\end{lstlisting}

\pagebreak

\subsection{No se puede asignar un nuevo pedido al mismo trabajador con menos de una hora de diferencia}
Este trigger se disparará si al intentar hacer una inserción en \texttt{Recoge},
la hora de la recogida no tiene al menos una hora de diferencia con las
anteriores.

Utilizaremos el cursor \texttt{cRecoge} para iterar en la tabla \texttt{Recoge};
la variable \texttt{diferencia} para calcular la diferencia entre la hora de la
nueva recogida y las que había en la tabla; y una variable \texttt{asignacion}
para guardar las tuplas que recoja el cursor.

Con un bucle iteraremos en la tabla extrayendo la hora de la \texttt{fecha} de
la tupla ya existente y calcular la diferencia con la nueva. Si esa diferencia
es menor que 1 y el \texttt{idTrabajador} y la edición coinciden, mostraremos
el error y se cancelará la inserción.

\begin{lstlisting}[language=sql]
CREATE TRIGGER recogida
	BEFORE INSERT ON Recoge
	FOR EACH ROW
DECLARE
	CURSOR cRecoge IS SELECT * FROM Recoge;
	diferencia NUMERIC;
	asignacion Recoge%ROWTYPE;
BEGIN
	OPEN cRecoge;
	FETCH cRecoge INTO asignacion;
	WHILE (cRecoge%FOUND) LOOP
		diferencia := EXTRACT(HOUR FROM :new.fecha - asignacion.fecha);

		IF(diferencia < 1 AND asignacion.idPersonal = :new.idPersonal AND asignacion.año = :new.año) THEN
			raise_application_error(-20600, :new.fecha || 'Un trabajador no puede serle asignado un pedido con menos de 1 hora de diferencia');
		END IF;

		FETCH cRecoge INTO asignacion;
	END LOOP;

	CLOSE cRecoge;
END;
\end{lstlisting}
