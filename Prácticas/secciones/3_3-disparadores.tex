\section{Disparadores}

\subsection{Un jugador no puede pertenecer a dos parejas distintas en la misma edición}
Este disparador afecta a la tabla \texttt{Inscrita} donde se almacenan las parejas
junto a la edición en la que participan.

Con el cursor \texttt{cJugador} recorremos la tabla primero comprobando que el
id de ninguno de los nuevos jugadores esté ya inscrito como pareja de otro jugador
que ya estuviese en la tabla; si ocurre esto informamos del error con el que estaría repetido.


\begin{lstlisting}[language=sql]
CREATE OR REPLACE TRIGGER jugador_unico
	BEFORE
	INSERT ON Inscrita
	FOR EACH ROW
DECLARE
	CURSOR cJugador IS SELECT * FROM Inscrita WHERE (:new.Año = Año);
	jugadores Inscrita%ROWTYPE;
BEGIN
	OPEN cJugador;
	FETCH cJugador INTO jugadores;
	WHILE (cJugador%FOUND) LOOP
		IF (:new.idJugador1 = jugadores.idJugador1 OR :new.IdJugador1 = jugadores.idJugador2)
			THEN raise_application_error(-20600, :new.idJugador1 || 'JUGADOR 1: No se puede pertenecer a más de una pareja en la misma edicion');
		END IF;

		IF (:new.idJugador2 = jugadores.idJugador1 OR :new.IdJugador2 = jugadores.idJugador2)
			THEN raise_application_error(-20600, :new.idJugador2 || 'JUGADOR 2: No se puede pertenecer a más de una pareja en la misma edicion');
		END IF;

		FETCH cJugador INTO jugadores;
	END LOOP;

	CLOSE cJugador;
END;
\end{lstlisting}

\subsection{No se pueden jugar más de dos partidos en la misma pista}

Según nuestro diseño del sistema de información este disparador necesitaba acceder
a atributos de varias tablas por lo que para simplificarlo hemos creado una vista:

\begin{lstlisting}[language=sql]
CREATE VIEW PistaPartidoFecha
AS (SELECT DISTINCT SeJuegaEn.idPista,SeJuegaEn.idPartido,Partido.fecha
	FROM SeJuegaEn, SeJuegaEn cjuega, Partido, Partido par
	WHERE(
		((Partido.fecha - par.fecha)<1 AND Partido.idPartido<>par.idPartido)
		AND
		(SeJuegaEn.idPista=cjuega.idPista AND SeJuegaEn.idPartido<>cjuega.idPartido)
		AND
		(Partido.idPartido=SeJuegaEn.idPartido)
));
\end{lstlisting}

Para el disparador necesitamos dos cursores: uno para coger las fechas de los partidos
(\texttt{cPartidoFecha}) y otro para recoger solo los partidos (\texttt{cPartido}),
y una variable donde guardemos el número de partidos que hay en la misma pista el mismo día.

Dentro de los bucles comprobaremos si en el caso de que el nuevo partido coincida en fecha
y pista con alguno de los que ya existen no sea el tercero que se jugara ese día, es decir,
si ya hay dos no se insertará en la tabla.

\begin{lstlisting}[language=sql]
CREATE OR REPLACE TRIGGER partido_unico
	BEFORE
	INSERT ON SeJuegaEn
	FOR EACH ROW
DECLARE
	CURSOR cPartidoFecha IS SELECT * FROM PistaPartidoFecha;
	CURSOR cPartido IS SELECT * FROM Partido;
	datos PistaPartidoFecha%ROWTYPE;
	datosPartido Partido%ROWTYPE;
	numPartidos number;
BEGIN
	OPEN cPartidoFecha;
	OPEN cPartido;
	SELECT (SELECT COUNT(*) FROM PistaPartidoFecha) INTO numPartidos FROM dual;
	FETCH cPartidoFecha INTO datos;
	FETCH cPartido INTO datosPartido;

	WHILE(cPartidoFecha%FOUND)LOOP
		WHILE(cPartido%FOUND)LOOP
			IF(numPartidos >= 2 AND :new.idPista = datos.idPista AND :new.idPartido <> datos.idPartido AND
				(:new.idPartido = datosPartido.idPartido AND (datosPartido.fecha-datos.fecha) < 1))
				THEN raise_application_error(-20600, :new.idPartido || 'No se jugar más de dos partidos en la misma pista en un mismo dia');
			END IF;

			FETCH cPartido INTO datosPartido;
		END LOOP;

		FETCH cPartidoFecha INTO datos;
	END LOOP;

	CLOSE cPartido;
	CLOSE cPartidoFecha;
END;
\end{lstlisting}

\subsection{No se puede modificar el dinero aportado por un colaborador o patrocinador si ya ha empezado el torneo}

\begin{lstlisting}[language=sql]
CREATE OR REPLACE TRIGGER dinero_tras_inicio_colab
	BEFORE UPDATE OF Dinero_aportado ON Colabora
	FOR EACH ROW
DECLARE
	CURSOR cPrimerPartido IS
	SELECT fecha FROM Partido WHERE fecha IN (SELECT MIN(fecha) FROM Partido
	WHERE (EXTRACT(YEAR FROM fecha) = :old.Año) GROUP BY to_char(fecha,'YYYY')) GROUP BY fecha;

	earliest DATE;
BEGIN
	OPEN cPrimerPartido;
	FETCH cPrimerPartido INTO earliest;

	IF (earliest < CURRENT_DATE)
		THEN raise_application_error(-20600, :new.Dinero_aportado || 'No se puede modificar el dinero aportado si ya ha comenzado el torneo');
	END IF;

	CLOSE cPrimerPartido;
END;
/

CREATE OR REPLACE TRIGGER dinero_tras_inicio_patr
	BEFORE UPDATE OF Dinero_aportado ON Patrocina
	FOR EACH ROW
DECLARE
	CURSOR cPrimerPartido IS
	SELECT fecha FROM Partido WHERE fecha IN (SELECT MIN(fecha) FROM Partido
	WHERE (EXTRACT(YEAR FROM fecha) = :old.Año) GROUP BY to_char(fecha,'YYYY')) GROUP BY fecha;

	earliest DATE;
BEGIN
	OPEN cPrimerPartido;
	FETCH cPrimerPartido INTO earliest;

	IF (earliest < CURRENT_DATE)
		THEN raise_application_error(-20600, :new.Dinero_aportado || 'No se puede modificar el dinero aportado si ya ha comenzado el torneo');
	END IF;

	CLOSE cPrimerPartido;
END;
/
\end{lstlisting}

\subsection{No se puede modificar el salario de un trabajador si ya ha realizado algún trabajo}

\begin{lstlisting}[language=sql]
CREATE OR REPLACE TRIGGER modificaSalario
	BEFORE UPDATE OF sueldo ON Trabaja
	FOR EACH ROW
DECLARE
	CURSOR fechas IS SELECT FechaInicio FROM Asigna WHERE (idPersonal = :new.idPersonal) ;
	registro DATE;
BEGIN
	OPEN fechas;
	FETCH fechas INTO registro;

	WHILE (fechas%FOUND)LOOP
		IF (sysdate > registro) THEN
			RAISE_APPLICATION_ERROR(-20600, :new.sueldo || '***EXCEPTION**** No se puede modificar el sueldo ya que ya ha empezado su trabajo');
		END IF;
		FETCH fechas INTO registro;
	END LOOP;

CLOSE fechas;
END;
/
\end{lstlisting}

\subsection{No se puede asignar un nuevo pedido al mismo trabajador con menos de una hora de diferencia}
\begin{lstlisting}[language=sql]
CREATE TRIGGER recogida
	BEFORE INSERT ON Recoge
	FOR EACH ROW
DECLARE
	CURSOR cRecoge IS SELECT * FROM Recoge;
	diferencia NUMERIC;
	asignacion Recoge%ROWTYPE;
BEGIN
	OPEN cRecoge;
	FETCH cRecoge INTO asignacion;
	WHILE (cRecoge%FOUND) LOOP
		diferencia := EXTRACT(HOUR FROM :new.fecha - asignacion.fecha);

		IF(diferencia < 1 AND asignacion.idPersonal = :new.idPersonal AND asignacion.año = :new.año) THEN
			raise_application_error(-20600, :new.fecha || 'Un trabajador no puede serle asignado un pedido con menos de 1 hora de diferencia');
		END IF;

		FETCH cRecoge INTO asignacion;
	END LOOP;

	CLOSE cRecoge;
END;
\end{lstlisting}
