\section{Descripción del sistema}
Torneo de padel.

Se nos requiere construir un sistema para la gestión de un torneo de pádel por parejas. Este torneo se celebra en ediciones de caracter anual. El sistema administrará tanto aspectos referidos exclusivamente al juego; como cuestiones técnicas respecto a su organización. Veámoslo.


En primer lugar, los participantes del torneo. Estos compiten por parejas y, dado que no existen distintas categorías de juego dentro del sistema, cada jugador sólo puede inscribirse una vez con una única pareja.

Las parejas tienen la libertad de escoger practicar bajo la tutela de un entrenador. Cada pareja solo puede tener un entrenador asignado en un momento dado, no siendo así para los entrenadores, que pueden dirigir a cuantas parejas deseen. 

Durante el torneo, las parejas se irán posicionando en un ránking que determina sus próximos enfrentamientos.


Previa realización del torneo se abre la venta de entradas online. Para realizar un pedido se requiere una cuenta de usuario en la página web oficial. El evento se reparte en varias etapas , cada cual con su propia entrada y precio, pero existen opciones como el pase de temporada para obtener un precio reducido. Las existencias de entradas son limitadas.

Los usuarios registrados en la página pueden realizar una única compra al día, y solo una compra puede estar activa en carrito. Cuando finaliza una compra, se les envía un recibo con toda la información pertinente; este se puede imprimir en casa, mostrar desde el teléfono móvil, o decir en taquilla el número identificador único a cada entrada.


Las contratas de los árbitros también se gestionan desde el sistema. Funciona bajo un sistema de mejor postor, en el que los árbitros reciben múltiples ofertas, que pueden aceptar, rechazar o contraofertar. 

El intercambio de ofertas entre postor y árbitro funciona tal que una oferta puede ser o bien aceptada o rechazada, acabando ahí; o bien ser contraofertada. El postor inicial es ahora quién se responsabiliza de aceptar o denegar esta contraoferta. Ahí termina el ciclo de vida de la oferta inicial: si el postor no está de acuerdo con la contraoferta recibida, se debe iniciar un proceso de oferta nuevo. 

Una vez cualquiera de las dos partes han aceptado o rechazado una oferta o contraoferta, la decisión es definitiva, y la oferta no puede ser modificada de ninguna manera.


Dentro de lo que es el torneo, existe una asignación de partidos a las pistas del recinto. Los partidos se juegan con un árbitro preasignado que, obviamente, ha aceptado una oferta para trabajar en el susodicho. Sin embargo, un árbitro solo puede pitar un partido diario.

Los partidos se asignan de forma previa a las pistas para poder publicar un horario con las correspondencias del día. La jornada del torneo dura 8 horas, y tras cada partido debe haber un periodo mínimo de 3 horas para la correcta limpieza y preparación de la pista y gradas. Por esto, en un mismo día no puede haber más de dos partidos en una misma pista.

Los resultados de los partidos se registrarán y harán públicos tras su finalización, con objeto de emplear estos datos en la elaboración del ránking.


El personal del evento se organizará por turnos delimitados en el día según horario y pista. Los turnos de un trabajador no deben solaparse con menos de una hora de diferencia para darle tiempo para abandonar el turno en la pista previa y llegar a la siguiente, además de un breve descanso. Un empleado no trabajará más de 8 horas en un día, y puede trabajar como máximo dos veces la misma pista en el día.

Una vez un empleado ha fichado para el día, no se puede modificar su salario previamente acordado.


Este evento cuenta con la colaboración y patrocinio de entidades externas a la organización. Este puede ser monetario o de material, y cuando una entidad se registra como patrocinador o colaborador, no se puede modificar durante la edición. Una vez el torneo ha comenzado, tampoco se les permite retirarse o modificar la cantidad monetaria aportada.


Finalmente, el material necesario para el torneo se suministra por pedidos por correo. Estos pedidos deben de ser recogidos por un empleado de la organización para firmar el albarán: todos los pedidos deben ser confirmados al llegar, y se recogen en una de las pistas del polideportivo. 

Los trabajadores de la pista durante el horario de llegada son los encargados de recibirlo, pero se debe notificar al trabajador con mínimo una hora de antelación. La asignación no es revocable: bajo ningún concepto se permite la cancelación de un pedido ya asignado a un trabajador.

Los pedidos son provistos por los patrocinadores, así que cada pedido corresponde únicamente a una entidad patrocinadora.


Estas son todas las características e indicaciones relevantes a este sistema de torneos de pádel. A continuación serán desglosadas en forma de requisitos y restricciones



