\subsection{Jugadores-Entrenadores}

\subsubsection{RF1.1: Insertar nuevo jugador en la base de datos}

\textbf{Entrada:} Agente externo: Jugador        Requisito de datos de entrada: RDE1.1

\textbf{BD:} Requisito de datos de escritura RDW1.1

\textbf{Salida:} Agente externo: ninguno

Requisito de datos de salida: ninguno

\textbf{RDE1.1:} Datos de entrada de jugador:
\begin{itemize}
	\item Nombre: cadena de caracteres (30)
	\item Apellidos: cadena de caracteres (60)
	\item Teléfono: cadena numérica (12, el primero puede ser +)
	\item E-mail: cadena de caracteres (50, debe contener @)
\end{itemize}
\textbf{RDW1.1:} Datos almacenados del jugador:
\begin{itemize}
	\item Nombre: cadena de caracteres (30)
	\item Apellidos: cadena de caracteres (60)
	\item Teléfono: cadena numérica (12, el primero puede ser +)
	\item E-mail: cadena de caracteres (50, debe contener @)
\end{itemize}


\subsubsection{RF1.2: Asignar entrenador a pareja en una edición del torneo}

\textbf{Entrada:} Agente externo: Jugador        Requisito de datos de entrada: RDE1.2

\textbf{BD:} Requisito de datos de escritura RDW1.2, requisito de datos de lectura RDR1.1

\textbf{Salida:} Agente externo: ninguno

Requisito de datos de salida: ninguno

\textbf{RDE1.2:} Datos de entrada de entrenador, la pareja y la edición:
\begin{itemize}
	\item Edición: número entero (4, corresponde al año)
\newline
	\item Nombre jugador 1: cadena de caracteres (30)
	\item Apellidos jugador 1: cadena de caracteres (60)
	\item Teléfono jugador 1: cadena numérica (12, el primero puede ser +)
	\item E-mail jugador 1l: cadena de caracteres (50, debe contener la @)
\newline
	\item Nombre jugador 2: cadena de caracteres (30)
	\item Apellidos jugador 2: cadena de caracteres (60)
	\item Teléfono jugador 2: cadena numérica (12, el primero puede ser +)
	\item E-mail jugador 2: cadena de caracteres (50)
\newline
	\item Nombre entrenador: cadena de caracteres (30)
	\item Apellidos entrenador: cadena de caracteres (60)
	\item Teléfono entrenador: cadena numérica (12, el primero puede ser +)
	\item E-mail entrenador: cadena de caracteres (50)
\end{itemize}

\textbf{RDW1.2:} Datos almacenados de la pareja:
\begin{itemize}
	\item Nombre jugador 1: cadena de caracteres (30)
	\item Apellidos jugador 1: cadena de caracteres (60)
	\item Teléfono jugador 1: cadena numérica (12, el primero puede ser +)
	\item E-mail jugador 1: cadena de caracteres (50, debe contener la @)
\newline
	\item Nombre jugador 2: cadena de caracteres (30)
	\item Apellidos jugador 2: cadena de caracteres (60)
	\item Teléfono jugador 2: cadena numérica (12, el primero puede ser  +)
	\item E-mail jugador 2: cadena de caracteres (50)
\newline
	\item Nombre entrenador: cadena de caracteres (30)
	\item Apellidos entrenador: cadena de caracteres (60)
	\item Teléfono entrenador: cadena numérica (12, el primero puede ser +)
	\item E-mail entrenador: cadena de caracteres (50)
\end{itemize}

\textbf{RDR1.1:} Datos de salida de la pareja:
\begin{itemize}
	\item Nombre jugador 1: cadena de caracteres (30)
	\item Apellidos jugador 1: cadena de caracteres (60)
	\item Teléfono jugador 1: cadena numérica (12, el primero puede ser +)
	\item E-mail jugador 1: cadena de caracteres (50, debe contener la @)
\newline
	\item Nombre jugador 2: cadena de caracteres (30)
	\item Apellidos jugador 2: cadena de caracteres (60)
	\item Teléfono jugador 2: cadena numérica (12, el primero puede ser  +)
	\item E-mail jugador 2: cadena de caracteres (50)
\newline
	\item Nombre entrenador: cadena de caracteres (30)
	\item Apellidos entrenador: cadena de caracteres (60)
	\item Teléfono entrenador: cadena numérica (12, el primero puede ser +)
	\item E-mail entrenador: cadena de caracteres (50)
\end{itemize}

\subsubsection{RF1.3: Consultar entrenadores y parejas que entrena en esa edición}

\textbf{Entrada:} Agente externo: Organizador    Requisito de datos de entrada: RDE1.3

\textbf{BD:} Requisito de datos de lectura RDR1.2

\textbf{Salida:} Agente externo: Organizador

Requisito de datos de salida: RDS1.1

\textbf{RDE1.3:} Datos de entrada de edición:
\begin{itemize}
	\item Edición: número entero (4, corresponde al año)
\end{itemize}

\textbf{RDR1.2:} Datos de lectura de pareja y entrenador:
\begin{itemize}
	\item Edición: número entero (4 cifras)
\newline
	\item Nombre jugador 1: cadena de caracteres (30)
	\item Apellidos jugador 1: cadena de caracteres (60)
	\item Teléfono jugador 1: cadena numérica (12, el primero puede ser +)
	\item E-mail jugador 1: cadena de caracteres (50, debe contener la @)
\newline
	\item Nombre jugador 2: cadena de caracteres (30)
	\item Apellidos jugador 2: cadena de caracteres (60)
	\item Teléfono jugador 2: cadena numérica (12, el primero puede ser  +)
	\item E-mail jugador 2: cadena de caracteres (50)
\newline
	\item Nombre entrenador: cadena de caracteres (30)
	\item Apellidos entrenador: cadena de caracteres (60)
	\item Teléfono entrenador: cadena numérica (12, el primero puede ser  +)
	\item E-mail entrenador: cadena de caracteres (50, debe contener la @)
\end{itemize}

\textbf{RDS1.1:} Datos de salida de pareja y entrenador:
\begin{itemize}
	\item Nombre jugador 1: cadena de caracteres (30)
	\item Apellidos jugador 1: cadena de caracteres (60)
	\item Teléfono jugador 1: cadena numérica (12, el primero puede ser +)
	\item E-mail jugador 1: cadena de caracteres (50, debe contener la @)
\newline
	\item Nombre jugador 2: cadena de caracteres (30)
	\item Apellidos jugador 2: cadena de caracteres (60)
	\item Teléfono jugador 2: cadena numérica (12, el primero puede ser  +)
	\item E-mail jugador 2: cadena de caracteres (50)
\end{itemize}

\subsubsection{RF1.4: Inscribir pareja de jugadores en una edición}

\textbf{Entrada:} Agente externo: Jugador        Requisito de datos de entrada: RDE1.4

\textbf{BD:} Requisito de datos de escritura RDW1.3, requisito de datos de lectura RDR1.3

\textbf{Salida:} Agente externo: ninguno

Requisito de datos de salida: ninguno

\textbf{RDE1.4:} Datos de entrada de la pareja:
\begin{itemize}
	\item Edición: número entero (4, corresponde al año)
\newline
	\item Nombre jugador 1: cadena de caracteres (30)
	\item Apellidos jugador 1: cadena de caracteres (60)
	\item Teléfono jugador 1: cadena numérica (12, el primero puede ser +)
	\item E-mail jugador 1: cadena de caracteres (50, debe contener la @)
\newline
	\item Nombre jugador 2: cadena de caracteres (30)
	\item Apellidos jugador 2: cadena de caracteres (60)
	\item Teléfono jugador 2: cadena numérica (12, el primero puede ser  +)
	\item E-mail jugador 2: cadena de caracteres (50)
\end{itemize}

\textbf{RDW1.3:} Datos almacenados del jugador:
\begin{itemize}
	\item Edición: número entero (4 cifras)
\newline
	\item Nombre jugador 1: cadena de caracteres (30)
	\item Apellidos jugador 1: cadena de caracteres (60)
	\item Teléfono jugador 1: cadena numérica (12, el primero puede ser +)
	\item E-mail jugador 1: cadena de caracteres (50, debe contener la @)
\newline
	\item Nombre jugador 2: cadena de caracteres (30)
	\item Apellidos jugador 2: cadena de caracteres (60)
	\item Teléfono jugador 2: cadena numérica (12, el primero puede ser  +)
	\item E-mail jugador 2: cadena de caracteres (50)
\end{itemize}

\textbf{RDR1.3:} Datos de salida del jugador:
\begin{itemize}
	\item Edición: número entero (4 cifras)
\newline
	\item Nombre jugador 1: cadena de caracteres (30)
	\item Apellidos jugador 1: cadena de caracteres (60)
	\item Teléfono jugador 1: cadena numérica (12, el primero puede ser +)
	\item E-mail jugador 1: cadena de caracteres (50, debe contener la @)
\newline
	\item Nombre jugador 2: cadena de caracteres (30)
	\item Apellidos jugador 2: cadena de caracteres (60)
	\item Teléfono jugador 2: cadena numérica (12, el primero puede ser  +)
	\item E-mail jugador 2: cadena de caracteres (50)
\end{itemize}


\subsubsection{RF1.5: Mostrar las parejas de jugadores por ranking en una edición}

\textbf{Entrada:} Agente externo: Organizador        Requisito de datos de entrada: RDE1.5

\textbf{BD:} Requisito de datos de lectura RDR1.4

\textbf{Salida:} Agente externo: Organizador

Requisito de datos de salida: RDS1.2

\textbf{RDE1.5:} Datos de entrada de ranking y edición:
\begin{itemize}
	\item Edición: número entero (4, corresponde al año)
	\item Cota superior: número entero
\end{itemize}

\textbf{RDR1.1:} Datos almacenados del jugador:
\begin{itemize}
	\item Edición: número entero (4 cifras)
	\item Posición ranking: número entero (hasta 2 cifras)
\newline
	\item Nombre jugador 1: cadena de caracteres (30)
	\item Apellidos jugador 1: cadena de caracteres (60)
	\item Teléfono jugador 1: cadena numérica (12, el primero puede ser +)
	\item E-mail jugador 1: cadena de caracteres (50, debe contener la @)
\newline
	\item Nombre jugador 2: cadena de caracteres (30)
	\item Apellidos jugador 2: cadena de caracteres (60)
	\item Teléfono jugador 2: cadena numérica (12, el primero puede ser  +)
	\item E-mail jugador 2: cadena de caracteres (50)
\end{itemize}

\textbf{RDS1.2:} Datos almacenados del jugador:
\begin{itemize}
	\item Posición de la pareja: número entero (hasta 2 cifras)
\newline
	\item Nombre jugador 1: cadena de caracteres (30)
	\item Apellidos jugador 1: cadena de caracteres (60)
	\item Teléfono jugador 1: cadena numérica (12, el primero puede ser +)
	\item E-mail jugador 1: cadena de caracteres (50, debe contener la @)
\newline
	\item Nombre jugador 2: cadena de caracteres (30)
	\item Apellidos jugador 2: cadena de caracteres (60)
	\item Teléfono jugador 2: cadena numérica (12, el primero puede ser  +)
	\item E-mail jugador 2: cadena de caracteres (50)
\end{itemize}

\subsubsection{RS1.1: Una pareja solo puede tener un entrenador en una edición}
\textbf{RF:} RF1.2

\textbf{RD(s):} RDE1.2,RDR1.1

\textbf{Descripción:} Si ya existe un entrenador asignado a la pareja, el nuevo entrenador no se registra y se da un mensaje de error

\subsubsection{RS1.2: Un jugador no puede pertenecer a dos parejas distintas en la misma edición}
\textbf{RF:} RF1.4

\textbf{RD(s):} RDE1.4, RDR1.3

\textbf{Descripción:} Si un jugador que ya pertenece a una pareja se intenta registrar con otra, la nueva pareja no se registra y se da un mensaje de error
\pagebreak
