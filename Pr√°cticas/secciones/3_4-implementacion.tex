\section{Implementación externa del Sistema de Información a la Base de Datos}
\subsection{Conexión}
La conexión a la base de datos la hacemos como vimos en el seminario 2, se crea
un objeto \texttt{Conection conn} en el que introducimos los datos necesarios para
la conexión:
\begin{lstlisting}[language=java]
Connection conn = DriverManager.getConnection("jdbc:oracle:thin:@//oracle0.ugr.es:1521/practbd.oracle0.ugr.es", "xDNI", "xDNI")
\end{lstlisting}

Esta conexión se realiza dentro de un try para que si falla salga la excepción de
SQL si lo que ha fallado es culpa de la base de datos, o una excepción de java por
si falla la creación del objeto para la conexión.

\subsection{Interfaz de usuario}
La interfaz de usuario se implementa con la entrada y salida de datos por
terminal.

Lo primero que ocurrirá es un mensaje para pedir el usuario y contraseña:
\begin{lstlisting}
Introduzca su usuario (que coincide con su password):
>	xDNI
Conectado a la base de datos
\end{lstlisting}

Una vez se realice la conexión nos aparecerá un menú inicial con el que poder
decidir dentro de que subsistema realizaremos transacciones:

\begin{lstlisting}
				--- MENU ---
1.Ediciones
2.Jugadores/Entrenadores
3.Pistas/Partidos
4.Patrocinadores/Colaboradores
5.Personal/Horarios
6.Materiales/Pedidos
7.Salir del programa
\end{lstlisting}

En cada una de esas opciones se nos abrirán nuevos submenús que muestren cada
una de las opciones disponibles.

\subsubsection{Ediciones}
\begin{lstlisting}
		--- Operaciones sobre Ediciones ---
1.Insertar nueva edición
2.Mostrar el total recaudado
3.Consultar las parejas de un entrenador
4.Mostrar el personal que no trabaja en una edición
5.Guardar cambios
6.Cancelar
\end{lstlisting}

En cada opción se nos solicitarán los datos necesarios para llamar a los
procedimientos de la base de datos.

En el caso de las opciones 2, 3 y 4 necesitamos usar un \texttt{CallableStatement}
para poder leer el parámetro de salida y mostrarlo por pantalla.

\subsubsection{Jugadores/Entrenadores}
\begin{lstlisting}
		--- Operaciones sobre jugadores y entrenadores ---
1.Insertar nuevo jugador
2.Insertar entrenador
3.Registrar pareja
4.Inscribir pareja en edición
5.Añadir entrenador a pareja
6.Añadir posicionamiento en el ranking
7.Guardar cambios
8.Cancelar
\end{lstlisting}

Como en el menú anterior se pedirán los datos que sean necesarios para cada
procedimiento.

En la opción 6, además se preguntará si se desea introducir la posición que ocupa
la pareja en el ranking. También existe la opción de añadir la posición después
si durante la realización del torneo esta posición cambiase.

\subsubsection{Pistas/Partidos}
\begin{lstlisting}
		--- Operaciones sobre pistas y partidos ---
1.Insertar pista
2.Insertar partido
3.Registrar resultado de partido
4.Guardar cambios
5.Cancelar
\end{lstlisting}

En el caso de registrar el partido nos saldrá la misma pregunta sobre añadir el
resultado nada más insertar el partido o después.

\subsubsection{Patrocinadores/Colaboradores}
\begin{lstlisting}
		--- Operaciones sobre colaboradores y patrocinadores ---
1.Insertar entidad
2.Registrar colaborador
3.Registrar patrocinador
4.Modificar dinero aportado
5.Guardar cambios
6.Cancelar
\end{lstlisting}

En modificar dinero aportado saldrá un menú más pequeño para decidir si modificar
sobre colaborador o sobre patrocinador, se podrán realizar varias modificaciones
hasta que decidamos o no guardar los cambios, entonces se volverá al menú del
subsistema para continuar la ejecución del programa.

\subsubsection{Personal/Horarios}
\begin{lstlisting}
		--- Operaciones sobre personal y horarios ---
1.Insertar personal
2.Registrar trabajador
3.Modificar salario
4.Guardar cambios
5.Cancelar
\end{lstlisting}

Este menú, no tiene ninguna particularidad, solo pedirá los datos para cada
procedimiento.

\subsubsection{Materiales/Pedidos}
\begin{lstlisting}
		--- Operaciones sobre materiales y pedidos ---
1.Insertar material
2.Insertar pedido
3.Asignar pedido a trabajador
4.Guardar cambios
5.Cancelar
\end{lstlisting}

Como en el caso anterior, este menú solo pide los datos para los procedimientos.

\subsubsection{Commit/Rollback}
En cada uno de los menús hay una opción para guardar que hará commit haciendo
efectivas las transacciones y otra para cancelar que hace un rollback anulando
los cambios que se han hecho.

En el caso en el que se modifica el dinero aportado por un colaborador o
patrocinador solo se anula lo insertado hasta antes de empezar a llamar al
procedimiento.

\subsection{Control de excepciones}
Para el control de excepciones cada llamada a una sentencia de SQL o que tenga
que ver con el control de la base de datos está dentro de un \texttt{try/except}
que gestionará si ha habido una excepción de SQL \texttt{SQLException} o una
generada por java: \texttt{Excepction}.
