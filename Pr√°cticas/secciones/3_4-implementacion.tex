\section{Implementación externa del Sistema de Información a la Base de Datos}
Para saber como compilar y ejecutar el programa ir al
\hyperref[cap:instrucciones]{ApéndiceB}

\subsection{Conexión}
La conexión a la base de datos la hacemos como vimos en el seminario 2, se crea
un objeto \texttt{Connection conn} en el que introducimos los datos necesarios para
la conexión:
\begin{lstlisting}[language=java]
Connection conn = DriverManager.getConnection("jdbc:oracle:thin:@//oracle0.ugr.es:1521/practbd.oracle0.ugr.es", "xDNI", "xDNI")
\end{lstlisting}

Esta conexión se realiza dentro de un try para que si falla salga la excepción de
SQL si lo que ha fallado es culpa de la base de datos, o una excepción de java
si falla la creación del objeto para la conexión.

\subsection{Interfaz de usuario}
La interfaz de usuario se implementa con la entrada y salida de datos por
terminal.

Lo primero que ocurrirá es un mensaje para pedir el usuario y contraseña:
\begin{lstlisting}
Introduzca su usuario (que coincide con su password):
>	xDNI
Conectado a la base de datos
\end{lstlisting}

Una vez nos conectemos a la base de datos, aparecerá un menú incial. Desde éste
podremos acceder a los distintos subsistemas o cerrar la conexión.

\begin{lstlisting}
				--- MENU ---
1.Ediciones
2.Jugadores/Entrenadores
3.Pistas/Partidos
4.Patrocinadores/Colaboradores
5.Personal/Horarios
6.Materiales/Pedidos
7.Salir del programa
\end{lstlisting}

En cada una de esas opciones se nos abrirán nuevos submenús en los que podremos
seleccionar las distintas funcionalidades.

\subsubsection{Ediciones}
\begin{lstlisting}
		--- Operaciones sobre Ediciones ---
1.Insertar nueva edición
2.Mostrar el total recaudado
3.Consultar las parejas de un entrenador
4.Mostrar el personal que no trabaja en una edición
5.Guardar cambios
6.Cancelar
\end{lstlisting}

En cada opción se nos solicitarán los datos necesarios para llamar a los
procedimientos de la base de datos.

En el caso de las opciones 2, 3 y 4 empleamos un \texttt{CallableStatement}
para poder leer el parámetro de salida y mostrarlo por pantalla.

\pagebreak

\subsubsection{Jugadores/Entrenadores}
\begin{lstlisting}
		--- Operaciones sobre jugadores y entrenadores ---
1.Insertar nuevo jugador
2.Insertar entrenador
3.Registrar pareja
4.Inscribir pareja en edición
5.Añadir entrenador a pareja
6.Añadir posicionamiento en el ranking
7.Guardar cambios
8.Cancelar
\end{lstlisting}

Como en el menú anterior, se pedirán los datos que sean necesarios para cada
procedimiento.

En la entrada de datos de la opción 4, existe la posibilidad de elegir si
introducimos o no la posición del ranking. Esto se debe a que existe la posibilidad
de inscribir la pareja antes del torneo, caso en el que no se conoce el ranking.
En cualquier caso, siempre tendremos la opción de añadirla más tarde desde la
opción 6.

\subsubsection{Pistas/Partidos}
\begin{lstlisting}
		--- Operaciones sobre pistas y partidos ---
1.Insertar pista
2.Insertar partido
3.Registrar resultado de partido
4.Guardar cambios
5.Cancelar
\end{lstlisting}

En el caso de registrar el partido se nos dará, de nuevo, la posibilidad de
insertar o no el resultado. En caso de no hacerlo, podremos actualizar la tabla
con el resultado desde la opción 3.

\subsubsection{Patrocinadores/Colaboradores}
\begin{lstlisting}
		--- Operaciones sobre colaboradores y patrocinadores ---
1.Insertar entidad
2.Registrar colaborador
3.Registrar patrocinador
4.Modificar dinero aportado
5.Guardar cambios
6.Cancelar
\end{lstlisting}

En \textit{Modificar dinero aportado} saldrá un menú más pequeño para elegir
si modificar sobre colaborador o sobre patrocinador: se podrán realizar varias
modificaciones hasta que decidamos o no guardar los cambios, entonces se volverá
al menú del subsistema para continuar la ejecución del programa.

\subsubsection{Personal/Horarios}
\begin{lstlisting}
		--- Operaciones sobre personal y horarios ---
1.Insertar personal
2.Registrar trabajador
3.Modificar salario
4.Asignar trabajador a una pista
5.Guardar cambios
6.Cancelar
\end{lstlisting}

Este menú no tiene ninguna particularidad, solo pedirá los datos para cada
procedimiento.

\pagebreak

\subsubsection{Materiales/Pedidos}
\begin{lstlisting}
		--- Operaciones sobre materiales y pedidos ---
1.Insertar material
2.Insertar pedido
3.Asignar pedido a trabajador
4.Introducir materiales a pedido
5.Guardar cambios
6.Cancelar
\end{lstlisting}

Para \textit{Introducir materiales a pedido}, una vez elegido el pedido,
seleccionaremos en bucle los distintos materiales y cantidades deseados. El bucle
puede detenerse tras cada material.

\subsubsection{Commit/Rollback}
En cada uno de los menús, menos en el principal, hay una opción para guardar,
que hará \texttt{commit} haciendo efectivas las transacciones; y otra para cancelar,
que hace un \texttt{rollback} anulando los cambios que se han hecho.

En el caso en el que se modifica el dinero aportado por un colaborador o
patrocinador solo se anula lo actualizado hasta antes de empezar a llamar al
procedimiento.

\subsection{Control de excepciones}
Para el control de excepciones cada llamada a una sentencia de SQL o que tenga
que ver con el control de la base de datos está dentro de un \texttt{try/except}
que gestionará si ha habido una excepción de SQL \texttt{SQLException} o una
generada por java: \texttt{Exception}.
