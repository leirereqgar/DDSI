\chapter{Instrucciones}\label{cap:instrucciones}
Para el uso de este programa, siga los siguientes pasos:
\begin{enumerate}
	\item Desde la \href{https://www.oracle.com/database/technologies/appdev/jdbc-ucp-19-7-c-downloads.html}{página de Oracle}, descargue el archivo ojdbc8.jar. Si selecciona el \href{https://download.oracle.com/otn-pub/otn_software/jdbc/197/ojdbc8.jar}{siguiente link}, la descarga se iniciará de forma automática.

	\item Ponga en el mismo directorio \texttt{ojdbc8.jar} y el archivo del programa, llamado \texttt{TPadel.java}
	\item Abra una terminal y ejecute \texttt{javac TPadel.java} para compilar
	\item A continuación, ejecute \texttt{java -cp ojdbc8.jar:. TPadel} para ejecutar el programa
	\item Introduzca su usuario (que es el mismo que su contraseña) de la base de datos oracle0.ugr.es. Este debería ser su DNI sin la letra y sustituyendo el primer dígito por una "x" minúscula
\end{enumerate}

También existe la opción de clonar \href{https://github.com/leirereqgar/DDSI}{nuestro repositorio de GitHub.} En tal caso:
\begin{enumerate}
	\item Ejecute \texttt{git clone https://github.com/leirereqgar/DDSI}
	\item Acceda al directorio /DDSI/Prácticas/codigo
	\item Abra una terminal y ejecute \texttt{javac TPadel.java} para compilar
	\item Ejecute \texttt{java -cp ojdbc8.jar:. TPadel} para ejecutar el programa
	\item Introduzca su usuario (que es el mismo que su contraseña) de la base de datos oracle0.ugr.es. Este debería ser su DNI sin la letra y sustituyendo el primer dígito por una "x" minúscula
\end{enumerate}
