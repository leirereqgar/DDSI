\chapter{Introducción}

En este seminario hemos implementado un sencillo SI basado en el procesado de pedidos. Contamos con tres tablas:
\begin{enumerate}
	\item Stock (Cproducto, Cantidad) 
	\item Pedido (Cpedido, Ccliente, Fecha\_pedido)
	\item Detalle\_Pedido (Cpedido, Cproducto, Cantidad)
\end{enumerate}

El sistema se conecta mediante JDBC a la base de datos oracle0.ugr.es. El programa principal, escrito en Java, cuenta con una interfaz simple de texto que permite seleccionar varias opciones hasta que decidamos finalizar la conexión a la base de datos.

El seminario se ha desarrollado de forma modular: tras consultar la documentación de JDBC de conexión a la base de datos, se programó el esqueleto que contiene los menús y finalmente se implementaron las sentencias SQL y comprobaciones necesarias. El reparto entre los colaboradores del equipo es como sigue:
\begin{enumerate}
	\item Javier Expósito: eliminación de Detalle\_pedido 
	\item Inés Nieto: conexión a la base de datos, inserción Pedido
	\item Laura Sánchez: eliminación de Pedido
	\item Leire Requena: menús, commits y rollback
	\item Clara Mª Romero: menús, inserción Detalle\_pedido
\end{enumerate}
