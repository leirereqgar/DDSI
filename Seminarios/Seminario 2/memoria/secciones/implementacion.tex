\chapter{Implementación y explicación del código}

\section{Menú}

La interfaz del programa se realiza desde la terminal, para ello se van indicando los pasos a seguir con la salida por pantalla y se espera la recogida de datos introducida por el usuario.

Se estructura con dos bucles while uno anidado dentro de otro con el que se comprueba la opción escogida por el usuario y se realiza la acción correspondiente en la base de datos. 

En el siguiente cuadro está la versión simplificada de esta parte del programa
\begin{lstlisting}[language=Java]
System.out.println("Introduzca su usuario (que coincide con su password): ");
String entrada;	
Scanner entradaEscaner = new Scanner (System.in);
entrada = entradaEscaner.nextLine();

// Aquí estaría la parte que conecta a la base de datos en el código original

System.out.println("\u001B[32mConectado a la base de datos\u001B[0m");

while( menu != 4 ){
	System.out.println("\n\u001B[34m\t\t--- MENU ---\u001B[0m");
	System.out.println("\u001B[36m1.Borrado y creacion de las tablas e insercion de tuplas\u001B[0m");
	System.out.println("\u001B[36m2.Dar de alta nuevo pedido\u001B[0m");
	System.out.println("\u001B[36m3.Borrar un pedido\u001B[0m");
	System.out.println("\u001B[36m4.Salir del programa y finalizar conexion\u001B[0m");

	entradaEscaner = new Scanner (System.in);
	menu = entradaEscaner.nextInt();

	switch(menu){
		case 1:
			// Código del borrado y la creación de las tablas
			break;

		case 2:
			System.out.println("\n\u001B[36m--- Introduzca los datos del pedido ---\u001B[0m");
			System.out.println("Código de pedido: ");
			entradaEscaner = new Scanner (System.in);
			int cpedido = entradaEscaner.nextInt();
	
			System.out.println("Código de cliente: ");
			entradaEscaner = new Scanner (System.in);
			int ccliente = entradaEscaner.nextInt();

			
			System.out.println("Fecha pedido (YYYY-MM-DD): ");
			entradaEscaner = new Scanner (System.in);
			String fecha = entradaEscaner.nextLine();
					  	
			//Comprobación e inserción en la tabla pedido

			int nuevo_pedido = -1;

			while(nuevo_pedido != 3 && nuevo_pedido != 4){
				System.out.println("\n\u001B[36m\t\t" + "--- NUEVO PEDIDO ---" + "\u001B[0m");
				System.out.println("\u001B[33m" + "1.Añadir detalle de producto" + "\u001B[0m");
				System.out.println("\u001B[33m" + "2.Eliminar todos los detalles de producto" + "\u001B[0m");
				System.out.println("\u001B[33m" + "3.Cancelar pedido" + "\u001B[0m");
				System.out.println("\u001B[33m" + "4.Finalizar pedido" + "\u001B[0m");

				entradaEscaner = new Scanner (System.in);
				nuevo_pedido = entradaEscaner.nextInt();

				switch(nuevo_pedido){
				case 1:
					System.out.println("Introduzca código de producto: ");
					entradaEscaner = new Scanner (System.in);
					int cproducto = entradaEscaner.nextInt();
					System.out.println("Introduzca cantidad: ");
					entradaEscaner = new Scanner (System.in);
					int cantidad = entradaEscaner.nextInt();

					// Comprobación de la cantidad e inserción en detalle pedido

					break;

					case 2:
						// Eliminación de los detalles del pedido

						break;

					case 3:
						//Anulación de los cambios hechos en las tablas
						System.out.println("\u001B[31m" + "No se han hecho efectivos los cambios" + "\u001B[0m");

						break;

					case 4:
						// Cambios efectivos en las tablas
						System.out.println("\u001B[32m" + "El pedido se ha guardado correctamente" + "\u001B[0m");

						break;

						default: System.out.println("\u001B[31m" + "Por favor, seleccione una opcion valida: " + "\u001B[0m");
				}
			}

			break;

			case 3:
				System.out.println("\nIntroduzca el código de pedido: ");
				entradaEscaner = new Scanner (System.in);
				cpedido = entradaEscaner.nextInt();
				
				// Eliminación del pedido y sus detalles correspondientes

				break;

			case 4:
				System.out.println("\u001B[31m" + "Cerrando la conexion..." + "\u001B[0m");
	
				break;

			default: System.out.println("\u001B[31m" + "Por favor, seleccione una opcion valida: " + "\u001B[0m");
			}
		}
}
\end{lstlisting}

\section{Borrado, creación e inserción de tuplas en las tablas}

Al inicio del programa hay que borrar las tablas para reiniciar la base de datos, para ello si al intentar ejecutar las sentencias de \texttt{DROP} da algún fallo porque la tabla no existía se avisa al usuario y se continúa con el programa. Además antes de eliminar la tabla se borran los datos con \texttt{DELETE}

\begin{lstlisting}[language=Java]
try{
	query = "DELETE Detalle_pedido";
	stmt  = conn.createStatement();
	stmt.executeUpdate(query);
	query = "DROP TABLE Detalle_pedido";
	stmt  = conn.createStatement();
	stmt.executeUpdate(query);
	System.out.println("Tabla Detalle_pedido dropeada");

} catch(Exception e){
	System.out.println("La tabla Detalle_pedido no existia");
}

try{
	query = "DELETE Pedido";
	stmt  = conn.createStatement();
	stmt.executeUpdate(query);

	query = "DROP TABLE Pedido";
	stmt  = conn.createStatement();
	stmt.executeUpdate(query);
	System.out.println("Tabla Pedido dropeada");

} catch(Exception e){
	System.out.println("La tabla Pedido no existia");
}

try{
	query = "DELETE Stock";
	stmt  = conn.createStatement();
	stmt.executeUpdate(query);

	query = "DROP TABLE Stock";
	stmt  = conn.createStatement();
	stmt.executeUpdate(query);
	System.out.println("Tabla Stock dropeada");

} catch(Exception e){
	System.out.println("La tabla Stock no existia");
}
\end{lstlisting}

Para la creación de las tablas usamos la sentencia \texttt{CREATE TABLE} indicando sus atributos, las claves primarias y las claves externas.

\begin{lstlisting}[language=Java]
//Crear tablas
query = "CREATE TABLE Stock(Cproducto INT NOT NULL, Cantidad INT, PRIMARY KEY (Cproducto))";
stmt  = conn.createStatement();
stmt.executeUpdate(query);

query = "CREATE TABLE Pedido (Cpedido INT NOT NULL, Ccliente INT, Fecha_pedido DATE, PRIMARY KEY (Cpedido))";
stmt  = conn.createStatement();
stmt.executeUpdate(query);

query = "CREATE TABLE Detalle_pedido (Cpedido INT NOT NULL, Cproducto INT NOT NULL, Cantidad int,FOREIGN KEY(Cpedido) REFERENCES Pedido(Cpedido),FOREIGN KEY(Cproducto) REFERENCES Stock(Cproducto),PRIMARY KEY(Cpedido,Cproducto))";
stmt  = conn.createStatement();
stmt.executeUpdate(query);
\end{lstlisting}

Para la inserción de tablas y el resto de sentencias de sql creamos una \texttt{String query} que reutilizaremos en todas ellas.

La inserción de las 10 tuplas predefinidas tiene que hacerse de uno en uno porque nuestra versión de JDBC no permite automatizarlo al dar algunos fallos en la sintaxis de sql.

\begin{lstlisting}[language=Java]
//Insertar tuplas
//String[] insert_query = new String[10];

query="INSERT INTO Stock(Cproducto, Cantidad) VALUES (1,34)";
stmt  = conn.createStatement();
stmt.executeUpdate(query);

query="INSERT INTO Stock(Cproducto, Cantidad) VALUES (2,66)";
stmt  = conn.createStatement();
stmt.executeUpdate(query);

query="INSERT INTO Stock(Cproducto, Cantidad) VALUES (3,12)";
stmt  = conn.createStatement();
stmt.executeUpdate(query)

query="INSERT INTO Stock(Cproducto, Cantidad) VALUES (4,46)";
stmt  = conn.createStatement();
stmt.executeUpdate(query);

query="INSERT INTO Stock(Cproducto, Cantidad) VALUES (5,2)";
stmt  = conn.createStatement();
stmt.executeUpdate(query);

query="INSERT INTO Stock(Cproducto, Cantidad) VALUES (6,34)";
stmt  = conn.createStatement();
stmt.executeUpdate(query);

query="INSERT INTO Stock(Cproducto, Cantidad) VALUES (7,21)";
stmt  = conn.createStatement();
stmt.executeUpdate(query);

query="INSERT INTO Stock(Cproducto, Cantidad) VALUES (8,17)";
stmt  = conn.createStatement();
stmt.executeUpdate(query);
query="INSERT INTO Stock(Cproducto, Cantidad) VALUES (9,50)";
stmt  = conn.createStatement();
stmt.executeUpdate(query);

query="INSERT INTO Stock(Cproducto, Cantidad) VALUES (10,71)";
stmt  = conn.createStatement();
stmt.executeUpdate(query);

conn.commit();
System.out.println("Datos insertados en la tabla Stock");
\end{lstlisting}

Al final de esta parte debemos hacer un commit para poder ver los cambios de la tabla de stock.

\section{Inserción en la tabla Pedido}

Antes de hacer ningún cambio creamos un savepoint para el commit o el rollback que finalizan esta sección.

Los datos se piden por pantalla de forma normal, excepto por la fecha que hay que comprobar que esté en el formato correcto (YYY-MM-DD) para poder insertarla en la tabla.

\begin{lstlisting}[language=Java]
//Creamos el savepoint al que volver si al final no se realiza el pedido
Savepoint save = conn.setSavepoint();

System.out.println("\n\u001B[36m--- Introduzca los datos del pedido ---\u001B[0m");
System.out.println("Código de pedido: ");
entradaEscaner = new Scanner (System.in);
int cpedido = entradaEscaner.nextInt();

System.out.println("Código de cliente: ");
entradaEscaner = new Scanner (System.in);
int ccliente = entradaEscaner.nextInt();

//COMPROBACIÓN DE LA FECHA	
boolean fecha_correcta = false;
boolean formato_correcto = false;
  								
while(!fecha_correcta){
	System.out.println("Fecha pedido (YYYY-MM-DD): ");
	entradaEscaner = new Scanner (System.in);
	String fecha = entradaEscaner.nextLine();
				  	
	//COMPROBACIÓN DEL FORMATO
	DateTimeFormatter formatter = DateTimeFormatter.ofPattern("yyyy-MM-dd");
	try{
		formatter.parse(fecha, LocalDate::from);
		formato_correcto = true;
	}
	catch(DateTimeParseException e){
		System.out.println("\u001B[31m" + "Formato incorrecto " + "\u001B[0m");
	}
	  	
	if(formato_correcto){
	  	//COMPROBACIÓN DE LA FECHA
	  	
	  	/*
	  	 * Recogemos el mes y la fecha de la cadena de caracteres que se ha pasado para ver si son correctos
	  	 */
	  	//mes
	  	String res = fecha.substring(5,7);
	  	int mes = Integer.parseInt(res);
									  	
	  	//día
		res = fecha.substring(8,10);
	  	int dia = Integer.parseInt(res);
									  	
		/*
		 * En el caso de que el mes sea febrero comprobamos que no se haya introducido un día mas allá del 28
		 */
	  	if(mes == 2){
	  		if(1 <= dia && dia <= 28){
				query = "INSERT INTO PEDIDO VALUES("+cpedido+","+ccliente + ",TO_DATE('"+fecha+"','YYYY-MM-DD'))";
		
				//"INSERT INTO Pedido VALUES(2,2,TO_DATE('2020-02-21','YYYY-MM-DD'))";
				stmt  = conn.createStatement();
				stmt.executeUpdate(query);
				System.out.println("\u001B[32mAñadiendo pedido \u001B[0m");
				fecha_correcta = true;
			}
		} 
		else{
			/*
			 * Se comprueba que en los meses con 31 días no se supere ese número
			 */
			if(mes%2 == 1){
				if(1 <= dia && dia <= 31){
					query = "INSERT INTO PEDIDO VALUES("+cpedido+","+ccliente + ",TO_DATE('"+fecha+"','YYYY-MM-DD'))";
									  				
					stmt  = conn.createStatement();
					stmt.executeUpdate(query);
					System.out.println("\u001B[32mAñadiendo pedido \u001B[0m");
					fecha_correcta = true;
				}
			}
			else{
				/*
				 * Comprobación para que el día no sea mayor que 30 en los meses con ese número de días
				 */
				if(1 <= dia && dia <= 30){
					query = "INSERT INTO PEDIDO VALUES("+cpedido+","+ccliente + ",TO_DATE('"+fecha+"','YYYY-MM-DD'))";	
					stmt  = conn.createStatement();
					stmt.executeUpdate(query);
  					System.out.println("\u001B[32mAñadiendo pedido \u001B[0m");
  					fecha_correcta = true;
  				}
  			}
		}
	}
}	
\end{lstlisting}

Como se puede ver la inserción es de la misma forma que en la tabla de \texttt{Stock} una vez que se ha comprobado que la fecha introducida es correcta y tiene el formato adecuado.

\subsection{Insertar los detalles de un pedido}
Como hemos visto al principio del capítulo dentro de la opción de añadir un pedido se pueden especificar o borrar los detalles correspondientes, además de hacer o no efectivo el pedido.

Para insertar los datos se pide el código del producto y la cantidad que se desea de él de forma normal:

\begin{lstlisting}[language=Java]
System.out.println("Introduzca código de producto: ");
entradaEscaner = new Scanner (System.in);
int cproducto = entradaEscaner.nextInt();

System.out.println("Introduzca cantidad: ");
entradaEscaner = new Scanner (System.in);
int cantidad = entradaEscaner.nextInt();

\end{lstlisting}

Después hay que comprobar si se dispone de la cantidad que se ha solicitado, para ello usamos un \texttt{SELECT} que nos indique la cantidad total del producto en stock y si es suficiente actualizamos el producto restandole las unidades que se han pedido.

Por último se inserta la tupla en \texttt{Detalle\_Pedido}

\begin{lstlisting}[language=Java]
stmt  = conn.createStatement();
ResultSet rset = stmt.executeQuery ("SELECT Cantidad FROM Stock WHERE Cproducto='" + cproducto + "'");

while(rset.next()){
	int cantidad_actual = rset.getInt(1);

	if(cantidad_actual >= cantidad){
		cantidad_actual -= cantidad;

		System.out.println("Cantidad disponible, procesando...");

		query="UPDATE Stock SET Cantidad=" + cantidad_actual + " WHERE Cproducto='" + cproducto + "'";
		stmt  = conn.createStatement();
		stmt.executeUpdate(query);

		System.out.println("Registrando pedido...");


		query="INSERT INTO Detalle_pedido(Cpedido, Cproducto, Cantidad) VALUES ((SELECT CPedido FROM Pedido WHERE Cpedido='" + cpedido + "'), (SELECT Cproducto FROM Stock WHERE Cproducto='" + cproducto + "')," + cantidad + ")";
		stmt  = conn.createStatement();
		stmt.executeUpdate(query);


	} else{
		System.out.println("Cantidad no disponible, faltan " + (cantidad-cantidad_actual) + " unidades");
	}
}
\end{lstlisting}

\subsection{Eliminación de los detalles del pedido}

Para borrar el detalle del pedido con el que estamos trabajando solo hay que indicar que borre la tupla si coincide con ese \texttt{cpedido}

\begin{lstlisting}[language=Java]
query="DELETE FROM Detalle\_pedido WHERE Cpedido= " + cpedido;
stmt.executeUpdate(query);
\end{lstlisting}

\subsection{Finalización del pedido}

Hay dos maneras de finalizar estas operaciones con el pedido, eliminar todos los cambios hechos o hacerlos efectivos.

Para eliminarlos simplemente hacemos un rollback para volver al savepoint que se ha creado antes de pedir los datos relativos al pedido:
\begin{lstlisting}[language=Java]
conn.rollback();
\end{lstlisting}

Y para hacerlos efectivos se llama a la función \texttt{commit}:

\begin{lstlisting}[language=Java]
conn.commit();
\end{lstlisting}

Ambas opciones nos sacan al menú principal para continuar con la aplicación.

\section{Borrado de un pedido y sus detalles}

Lo primero que hacemos es pedir el código del pedido que vamos a borrar:

\begin{lstlisting}[language=Java]
System.out.println("\nIntroduzca el código de pedido a eliminar: ");
entradaEscaner = new Scanner(System.in);
cpedido = entradaEscaner.nextInt();
\end{lstlisting}

Una vez tenemos el \texttt{cpedido} tenemos que comprobar que en la tabla existe un pedido con ese código, para ello hay que explicitar nuestro tipo de Result Statement para que perminta iterar hacia delante y hacia atrás. Por defecto, los Result Statement solo pueden iterarse hacia delante.

Si el primer resultado es nulo lanzamos un mensaje de error y salimos de esta opción; si es nulo deshacemos la iteración (\texttt{conn.previous()}) y continuamos el programa normalmente. Tendremos una variable bool para imprimir o no el éxito al borrar el pedido.

Tenemos en cuenta también que si se borra el pedido hay que actualizar la tabla stock para devolver la cantidad de productos indicada en Detalle\_pedido.

\begin{lstlisting}[language=Java]
Boolean existe = true;

stmt = conn.createStatement( ResultSet.TYPE_SCROLL_INSENSITIVE,ResultSet.CONCUR_READ_ONLY);
ResultSet cproducto1 = stmt.executeQuery("SELECT Cproducto,Cantidad FROM Detalle_pedido WHERE Cpedido='"+ cpedido + "'");

if (cproducto1.next() == false) {
	System.out.println("El pedido introducido no existe");
	existe = false;
} else {
	cproducto1.previous();
}

while (cproducto1.next()) {
	int cproducto2 = cproducto1.getInt(1);
	int ccantidad = cproducto1.getInt(2);
	int cantidad_total = 0;

	System.out.println("El pedido introducido corresponde a: ");
	System.out.println("  Código del producto: " + cproducto2);
	System.out.println("  Cantidad: " + ccantidad);

	//Actualizar la cantidad
	stmt  = conn.createStatement();
	ResultSet rset_stock = stmt.executeQuery ("SELECT Cantidad FROM Stock WHERE Cproducto='" + cproducto2 + "'");

	while(rset_stock.next()){
		cantidad_total = rset_stock.getInt(1) + ccantidad;
	}

	query = "UPDATE Stock SET Cantidad=" + cantidad_total + " WHERE Cproducto='" + cproducto2+ "'";
	stmt.executeUpdate(query);

}


//Borrado de tuplas
query = "DELETE FROM DETALLE_PEDIDO WHERE Cpedido = " + cpedido;
stmt = conn.createStatement();
stmt.executeUpdate(query);

query = "DELETE FROM PEDIDO WHERE Cpedido = " + cpedido;
stmt = conn.createStatement();
stmt.executeUpdate(query);

if(existe){
	conn.commit();
	System.out.println("\u001B[32m" + "El pedido se ha eliminado correctamente" + "\u001B[0m");
}
\end{lstlisting}


\section{Finalización del programa}

Cuando se selecciona esta opción se imprime un mensaje que indica que se va a cerrar la conexión con la base de datos y salimos del bucle para cerrar la conexión con \texttt{conn.close()}
