\chapter{Funcionalidades}
\section{Contabilidad (accounting)}
Proporciona características múltiples tales como contabilidad analítica con varios puntos clave, contabilidad de presupuesto multidimensional,informes genéricos y facturación

\section{Gestión de efectivo basada en simulación (Simulation-Based Cash)}
Permite planificar futuras transacciones en efectivo basadas en carteras de pedidos / facturas actuales o ingresar manualmente transacciones provisionales que se esperan en el futuro.

\section{Multi-Currency}
ERP5 Finance está diseñado para admitir múltiples monedas.El soporte multidivisa permite un seguimiento preciso de los saldos de clientes o proveedores al tiempo que garantiza una contabilidad precisa.

\section{Multi-Entity (entidad múltiple)}
Varias entidades o empresas pueden utilizar un único sistema para su contabilidad. Los derechos de seguridad avanzados impiden que una entidad acceda a las transacciones de otra entidad si es necesario

\section{Product Data Management(PDM)}
Proporciona una descripción transparente de productos.Además proporciona una solución para administrar productos, compra, venta, precios, dimensiones como tamaños, códigos de barras, etc.

\section{Inmobilidad}
Da la posibilidad de realizar inmovilizaciones y amortizaciones contables en ERP5

\section{Trade  (Comercio)}
Proporciona la gestión de los procesos de compra y venta: oferta, pedido, embalaje, entrega y condiciones comerciales

\section{Presupuesto}
Proporciona la creación ,el planteamiento y la administración de presupuestos

\section{Planeamiento de requisitos de material}
Proporciona todos los elementos necesarios para facilitar la producción, forneciendo los costes de los materiales, informes de la producción y el planeamiento de los requisitos materiales necesarios así como la cadena de proveedores

\section{Proyecto}
Proporciona la ayuda para el planteamiento del proyecto así como un análisis de los costes

\section{Document Management System(DMS)}
Es centralizado en los formatos de datos abiertos, pero también usa los formatos más conocidos como “.doc” o “.xls”

\section{Customer Relation Management (CRM)}
Incluye características para seguir el desarrollo de los clientes, la relación entre ellos y otras organizaciones, así como administrar llamadas telefónicas y email, oportunidades de ventas y también un sistema de peticiones de quejas y ayudas.
