\chapter{Ventajas e inconvenientes de ERP5}
\section{Ventajas}

\begin{enumerate}
	\item ERP5 se queda con las partes comunes de los 5 módulos que lo forman, haciéndolo así más versátil.
	\item Posee una interfaz fácil y clara de usar, que hace más sencillo su uso al cliente.
	\item Facilita el trabajo en equipo: 
	\begin{enumerate}
		\item Es multilingüe, lo que le hace ser aceptada por diferentes empresas y poder trabajar de forma internacional.
		\item Posee un motor de sincronización integrado basado en SyncML que le permite la implementación en sitios remotos con conectividad de red no confiable o permitir a los usuarios móviles traer un sistema ERP5 completo en su computadora portátil y más tarde sincronizarlo 
	\end{enumerate} 
	\item La información se deposita en una única base de datos, dándole consistencia al sistema
	\item Tiene el modelo de seguridad 5A: son 5 roles genéricos que nos permite que cada rol tenga sus permisos y, al dar de alta a un nuevo personal, no tengamos que fijar los permisos necesarios, sino incluirlo en su respectivo rol.
	\item Cobertura completa: ERP5 cubre contabilidad, gestión de relaciones con los clientes, comercio, gestión de almacenes, envío, facturación, gestión de recursos humanos, diseño de productos, producción y gestión de proyectos.
	\item Tiene un modelo de negocio unificado, es decir, todas las características desarrolladas para un módulo (ex. contabilidad) están disponibles en otro módulo (ex. gestión de almacenes). 
	\item Una gran ventaja es que podemos llevarnos nuestro software a cualquier lugar, ya que existe una aplicación diseñada para teléfonos móviles.
\end{enumerate}

\section{Inconvenientes}
\begin{enumerate}
	\item Su personalización es muy costosa → usted mismo podría ser un desarrollador de ERP5, pero la curva de aprendizaje es bastante empinada y nos llevaría algo más de un año. Por ello, lo común es que le pidamos a Nexedi (el encargado de mantenimiento) que nos implemente una aplicación de acuerdo a nuestras necesidades. Por lo general, el presupuesto empieza en 50.000€ y puede crecer hasta varios millones, según la aplicación.
	\item Ocupa demasiado espacio
	\item Está limitada su instalación en varios sistemas operativos, como por ejemplo Manjaro no es posible su instalación → solución: uso de una máquina virtual.
\end{enumerate}