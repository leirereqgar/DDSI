\chapter{Empresas que han implementado ERP5 con éxito}
\section{Grupo GKR} 
Líder en China en la industria del caucho. A lo largo de 13 años han creado nuevas plantas, cada una con su propia base de datos para los movimientos internos, ventas… Por lo tanto se decidió implementar un solo ERP5 para unificarlo todo.

\section{SANKEI}
Es una compañía de pesticidas de Japón. Gracias a la implementación de ERP5 ha sido capaces de integrar sus 6 webs y mejorar sus procesos que antes manejaban varios programas para la producción, orden, venta y almacén del producto. La integración fue posible por la habilidad de ERP5 de adaptar los requerimientos de la empresa.

\section{Zawee}
Es una compañía de moda localizada en París que lleva su tanto su inventario como su tienda online con ERP5. Según crecía la empresa, en sus tiendas físicas era difícil llevar el inventario teniendo en cuenta la variedad de tamaños, colores y modelos. 

Marcar como disponibles productos agotados da una imagen negativa de la empresa, así que se optó por ERP5. Siendo una pequeña empresa con 3 empleados, y aún sin experiencia en ERPs,  pronto pudieron implementar ERP5 a su negocio. Esto no sólo les ayudó con el inventario, sino que además se lanzaron a la venta online.

\section{Aide et Action}
Es una organización no gubernamental centrada en que la educación sea accesible para todos. Con sede en Ginebra Suiza, la organización tiene más de 1000 empleados y 120 proyectos en todo el mundo. La implementación de ERP5 se ha centrado en mejorar la transparencia y calidad de los reportes financieros e incrementar la productividad de sus equipos de contabilidad. Han implementado una configuración basada en la nube que cubre su contabilidad general y analítica y la gestión de los presupuestos de 30 entidades en 22 países distintos.

\section{WIDELIN}
Software open source donde se ha implementado ERP5 para poder monitorizar y mantener las turbinas de los aerogeneradores. La necesidad de mantenimiento se calculará de forma preventiva y por machine learning. En Alemania hay 450 turbinas perfectamente conectadas en tiempo real.

\section{Airbus DEFENSE \& SPACE}
En concreto, Airbus DS Geo GmbH, tiene la propiedad del radar TerraSAR-X para la observación de la tierra por satélite. El principal plan era vender imágenes por satélite, en principio se desarrolló en oracle pero al necesitar más flexibilidad en la personalización se implementó ERP5. La implementación se centra en poder personalizar las órdenes de los clientes y poder catalogar imágenes ya existentes con las nuevas que entran. También incluye una herramienta de seguridad.

\section{Universidad de Dresden}
Esta universidad diseño un MOOC(Masive Open Online Course), un curso online disponible para todo el mundo. El curso se diseña para una nueva forma de enseñar a través de la publicación de apuntes con licencia CC gracias al uso de ERP5 para apoyar el control de las lecturas y facilitar el contacto y colaboración con compañeros reales.

\section{Mitsubishi Motors}
La distribuidora rusa de Mitsubishi necesitaba un sistema de control de todos los contratos de terceros. Existía un procedimiento offline basado en los requerimientos legales entre los contratos y las empresas; pero este sistema que requería apoyo tanto de e-mails como de trabajo de oficina no era ni eficiente ni tenía trazabilidad alguna - Así que se optó por ERP5: se creó un módulo basado en toma de decisiones, en el que los empleados no necesitaban conocimientos técnicos del ERP, sólo aprobar o rechazar decisiones. También se implementó un sistema colaborativo de creación de documentos y control de versiones.